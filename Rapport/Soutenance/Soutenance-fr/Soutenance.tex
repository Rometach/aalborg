\documentclass[10pt]{beamer}

\usepackage[francais]{babel}
\usepackage[utf8x]{inputenc}
\usepackage[T1]{fontenc}
\usepackage{bookman}
\usepackage{amsmath}
\usepackage{amscd}
\usepackage{amssymb}
\usepackage{amsthm}
\usepackage{latexsym}
\usepackage{graphicx}
\usepackage{color}
\usepackage{xcolor}
\usepackage{calc}
\usepackage{setspace}
\usepackage[boxruled,vlined,french]{algorithm2e}
\usepackage{times}
\usepackage{tikz}
\usepackage{verbatim}

%\setlength{\voffset}{-3.75cm}
%\setlength{\hoffset}{-2.6cm}
%\setlength{\oddsidemargin}{2.75cm}
%\setlength{\topmargin}{2in}
%\setlength{\headheight}{0in}
%\setlength{\headsep}{0in}
%\setlength{\topskip}{0in}
\setlength{\parindent}{0cm}
%\setlength{\parskip}{1ex plus0.4ex minus0.2ex}
%\setlength{\textwidth}{16.25cm}
%\setlength{\textheight}{21cm}
%\renewcommand{\baselinestretch}{1.5}
\flushbottom
\setcounter{page}{1}
\setcounter{tocdepth}{1}

\SetKw{Edb}{Effet de bord}
\SetKw{Et}{et}
\SetKw{Ou}{ou}
\SetKw{De}{de}
\SetKw{A}{à}
\SetKwBlock{Debut}{Début}{Fin}
\SetKwIF{Si}{SinonSi}{Sinon}{Si}{alors}{Sinon si}{Sinon}{FinSi}
\SetKwFor{Pour}{Pour}{faire}{FinPour}
\SetKwFor{PourTout}{Pour tout}{faire}{FinPour}
\SetKwFor{TantQue}{Tant que}{faire}{FinTantQue}
\SetKw{Retour}{retourner}

\definecolor{bleuclair}{rgb}{0.2,0.9,8}
\definecolor{rose}{rgb}{1,0.5,0.7}
\definecolor{orange}{rgb}{1,0.6,0}


\newcommand{\guill}[1]{«~#1~»}

\usetikzlibrary{arrows,shapes}



\usetheme{Singapore}

\AtBeginSection[]
{
  \begin{frame}<beamer>
    \frametitle{Plan}
    \tableofcontents[currentsection]
  \end{frame}
}

\defbeamertemplate*{footline}{infolines theme}
{
 \hbox{
  \begin{beamercolorbox}[wd=.22\paperwidth,ht=2.25ex,dp=1ex,center]{author in head/foot}%
    \insertshortauthor%
  \end{beamercolorbox}%
  \begin{beamercolorbox}[wd=.57\paperwidth,ht=2.25ex,dp=1ex,center]{title in head/foot}%
    \insertshorttitle%
  \end{beamercolorbox}%
  \begin{beamercolorbox}[wd=.13\paperwidth,ht=2.25ex,dp=1ex,center]{date in head/foot}%
    \insertshortdate{}%
  \end{beamercolorbox}%
  \begin{beamercolorbox}[wd=.08\paperwidth,ht=2.25ex,dp=1ex,center]{author in head/foot}%
    \insertframenumber/\inserttotalframenumber%
  \end{beamercolorbox}}%
}



\title[Analyse de suites d'accords de jazz]{\large Soutenance de stage \\ \Large Analyse de suites d'accords de jazz}
\institute{ENS de Lyon, M1 Informatique fondamentale}

\author[R. Versaevel]{\normalsize Romain \textsc{Versaevel}\newline \\ \normalsize Encadré par Pr. David \textsc{Meredith}}

\date{\today}

\begin{document}


\begin{frame}
\titlepage
\end{frame}

\begin{frame}{Plan}
\tableofcontents
\end{frame}

\section{Motivation, modèle}
\subsection{Motivation, modèle}

\begin{frame}{Motivation}
\begin{figure}
\centering
\includegraphics[width = 9cm]{images/achildisborn.jpg}
\end{figure} \bigskip \pause

\begin{block}{\textbf{A Child Is Born: }}
$B\flat M7;~E\flat m;~B\flat M7;~E\flat m6;~B\flat M9;~E\flat m;~A~halfdim7;~D 7\#9\dots$
\end{block}

\end{frame}

\begin{frame}{Accords}
\begin{block}{Définition}
Un accord est un ensemble d'au moins trois notes.
\end{block} \bigskip \pause

\begin{figure}
\centering
\includegraphics[height = 1.4cm]{images/chordletter.jpg} ~~~~~~~~~~~~
\includegraphics[height = 1.4cm]{images/chordstaff.jpg}
\end{figure} \bigskip

\uncover<3->{
\begin{center}
\Huge
\only<1,2,3>{\begin{tabular}{c}
$B\flat M7$
\end{tabular}}
\only<4>{\begin{tabular}{ccc}
$B\flat$ & & $M7$
\end{tabular}}
\end{center}}
\end{frame}




\begin{frame}{Modèle mathématique}
Les notes sont des nombres. \bigskip \pause

\begin{tabular}{|c|c|c|c|c|c|c|c|c|c|c|c|}
\hline
%$A$&$A\#/B\flat$&$B$&$C$&$C\#/D\flat$&$D$&$D\#/E\flat$&$E$&$F$&$F\#/G\flat$&$G$ \\ \hline
%La&&Si&Do&&Ré&&Mi&Fa&&Sol& \\ \hline
%$A$&$A\#$&$B$&$C$&$C\#$&$D$&$D\#$&$E$&$F$&$F\#$&$G$&$G\#$ \\
&$A\#$&&&$C\#$&&$D\#$&&&$F\#$&&$G\#$ \\
$A$&$=$&$B$&$C$&$=$&$D$&$=$&$E$&$F$&$=$&$G$&$=$ \\
&$B\flat$&&&$D\flat$&&$E\flat$&&&$G\flat$&&$A\flat$ \\ \hline
0&1&2&3&4&5&6&7&8&9&10&11 \\ \hline
\end{tabular} \bigskip \pause

\begin{center}
\LARGE\begin{tabular}{ccc}
$B\flat$ & & $M7$ \uncover<4>{\\
$1$ & $+$ & $\{0;4;7;10\}$}
\end{tabular}
\end{center}
\end{frame}

\section{Algorithmes de compression}

\subsection{LZ77}

\begin{frame}{Lempel-Ziv 77}
\pause
\begin{figure}
\centering
\includegraphics[height = 4cm]{images/lempel.jpg} \hspace{0.5cm} \pause
\includegraphics[height = 4cm]{images/ziv.jpg} \hspace{0.5cm} \pause
\includegraphics[height = 4cm]{images/77.jpg}
\end{figure}
\end{frame}

\begin{frame}{LZ77: un aperçu}
\begin{block}{Entrée}
$\texttt{I}=ABCABCABD$
\end{block}

\bigskip \bigskip

\begin{block}{Sortie}
$(0,0,A)$,
$(0,0,B)$,
$(0,0,C)$,
$(3,5,D)$
\end{block}
\end{frame}


\subsection{Diagonal patterns}

\begin{frame}{Motifs diagonaux (1)}
\begin{figure}
\centering
\only<1>{\includegraphics[width = 6cm]{images/diagonals0.jpg}}
\only<2>{\includegraphics[width = 6cm]{images/diagonals1.jpg}}
\only<3>{\includegraphics[width = 6cm]{images/diagonals2.jpg}}
\end{figure}
\end{frame}


\begin{frame}{Motifs diagonaux (2)}

\begin{center}
\begin{tabular}{|ll|}
\hline
\multicolumn{2}{|l|}{\textbf{Algorithme}} \\ \hline & \\
1& Lister les motifs \\
2& Trouver une couverture \\ & \\ \hline
\end{tabular}
\end{center}

\end{frame}

\begin{frame}{Motifs diagonaux (3)}
\begin{block}{Entrée}
$\texttt{I}=
B;D 7;G;B\flat 7;E\flat;A m7;D 7;G;B\flat 7;E\flat;F\# 7;B;F m7;B\flat 7;E\flat$
\end{block} \pause

\begin{block}{Motifs}
\begin{itemize}
\item[$\triangleright$] $B$ --- $\{0;11\}$;
\item[$\triangleright$] $B\flat 7;~E\flat$ --- $\{3;8;13\}$;
\item[$\triangleright$] $D7;~G;~B\flat 7;~E\flat$ --- $\{1;6\}$; \pause
\item[$\triangleright$] $D7$ --- $\{1;6\}$, $G$ --- $\{2;7\}$\dots
\end{itemize}
\end{block} \pause

\begin{block}{Couverture}
\begin{itemize}
\item[$\triangleright$] $B$ --- $\{0;11\}$;
\item[$\triangleright$] $B\flat 7;~E\flat$ --- $\{13\}$;
\item[$\triangleright$] $D7;~G;~B\flat 7;~E\flat$ --- $\{1;6\}$;
\item[$\triangleright$] $Am7$ --- $\{5\}$, $F\#7$ --- $\{10\}$, $Fm7$ --- $\{12\}$
\end{itemize}
\end{block}
\end{frame}

\begin{frame}{Motifs diagonaux (4)}
\only<1,2>{\begin{tabular}{ccccccccccccccc}
\rotatebox{90}{$B$} & \rotatebox{90}{$D 7$} & \rotatebox{90}{$G$} & \rotatebox{90}{$B\flat 7$} & \rotatebox{90}{$E\flat$} & \rotatebox{90}{$A m7$} & \rotatebox{90}{$D 7$} & \rotatebox{90}{$G$} & \rotatebox{90}{$B\flat 7$} & \rotatebox{90}{$E\flat$} & \rotatebox{90}{$F\# 7$} & \rotatebox{90}{$B$} & \rotatebox{90}{$F m7$} & \rotatebox{90}{$B\flat 7$} & \rotatebox{90}{$E\flat$} \\ \hline \uncover<2>{\\
\rotatebox{90}{$B$} &&&&&&&&&&& \rotatebox{90}{$B$} \\
& \rotatebox{90}{$D 7$} & \rotatebox{90}{$G$} & \rotatebox{90}{$B\flat 7$} & \rotatebox{90}{$E\flat$} && \rotatebox{90}{$D 7$} & \rotatebox{90}{$G$} & \rotatebox{90}{$B\flat 7$} & \rotatebox{90}{$E\flat$} \\
&&&&& \rotatebox{90}{$A m7$} \\
&&&&&&&&&& \rotatebox{90}{$F\# 7$} \\
&&&&&&&&&&&& \rotatebox{90}{$F m7$} \\
&&&&&&&&&&&&& \rotatebox{90}{$B\flat 7$} & \rotatebox{90}{$E\flat$}}
\end{tabular}}

\only<3>{\begin{tabular}{ccccccccccccccc}
\color{bleuclair} \rotatebox{90}{$B$} & \color{rose} \rotatebox{90}{$D 7$} & \color{rose} \rotatebox{90}{$G$} & \color{rose} \rotatebox{90}{$B\flat 7$} & \color{rose} \rotatebox{90}{$E\flat$} & \color{green} \rotatebox{90}{$A m7$} & \color{rose} \rotatebox{90}{$D 7$} & \color{rose} \rotatebox{90}{$G$} & \color{rose} \rotatebox{90}{$B\flat 7$} & \color{rose} \rotatebox{90}{$E\flat$} & \color{red} \rotatebox{90}{$F\# 7$} & \color{bleuclair} \rotatebox{90}{$B$} & \color{blue} \rotatebox{90}{$F m7$} & \color{orange} \rotatebox{90}{$B\flat 7$} & \color{orange} \rotatebox{90}{$E\flat$} \\ \hline \\
\color{bleuclair} \rotatebox{90}{$B$} &&&&&&&&&&& \color{bleuclair} \rotatebox{90}{$B$} \\
& \color{rose} \rotatebox{90}{$D 7$} & \color{rose} \rotatebox{90}{$G$} & \color{rose} \rotatebox{90}{$B\flat 7$} & \color{rose} \rotatebox{90}{$E\flat$} && \color{rose} \rotatebox{90}{$D 7$} & \color{rose} \rotatebox{90}{$G$} & \color{rose} \rotatebox{90}{$B\flat 7$} & \color{rose} \rotatebox{90}{$E\flat$} \\
&&&&& \color{green} \rotatebox{90}{$A m7$} \\
&&&&&&&&&& \color{red} \rotatebox{90}{$F\# 7$} \\
&&&&&&&&&&&& \color{blue} \rotatebox{90}{$F m7$} \\
&&&&&&&&&&&&& \color{orange} \rotatebox{90}{$B\flat 7$} & \color{orange} \rotatebox{90}{$E\flat$}
\end{tabular}}
\end{frame}


\section{Mesures de similarité}
\subsection{Mesures de similarité}

\begin{frame}{Vers une compression \emph{avec perte}}
\begin{align*}
\Large
C & = C' \\
\uncover<2> {
& \downarrow \\
C &\sim C'}
\end{align*}
\end{frame}


\begin{frame}{Toutes les mesures}
\begin{itemize}
\item[$\triangleright$] égalité de la note fondamentale;
\item[$\triangleright$] équivalence par transposition;
\item[$\triangleright$] équivalence des \emph{PCS-Prime};
\only<2>{\color{blue}}
\item[$\triangleright$] F1-score;
\item[$\triangleright$] \emph{similarity index} d'Isaacson;
\item[$\triangleright$] mesure de Lewin;
\item[$\triangleright$] mesure de Morris;
\item[$\triangleright$] mesure de Rahn;
\item[$\triangleright$] mesure de Teitelbaum.
\end{itemize}
\end{frame}



\section{Résultats}

\begin{frame}{Évaluation}
\begin{block}{Facteur de compression}
\Large \uncover<2>{
$\frac{|\text{Entrée}|}{|\text{Sortie}|}$}
\end{block}

\bigskip

\begin{block}{Facteur de conservation}
\Large \uncover<2>{
$\frac{|\{i~|~\textsc{decompress}(\textsc{compress}(\text{Entrée}))[i] = \text{Entrée}[i]\}|}{|\text{Entrée}|}$ }
\end{block}
\end{frame}


\subsection{Comparaison des mesures utilisées}

\begin{frame}{Comparaison des mesures (1)}

\begin{figure}
\centering
\includegraphics[width = 7cm]{images/RfC77.jpg}
\end{figure}
\end{frame}


\begin{frame}{Comparaison des mesures (2)}
\only<1> {
\begin{figure}
\centering
\includegraphics[width = 5cm]{images/F1Diag.jpg} \hspace{0.5cm}
\includegraphics[width = 5cm]{images/IsaacsonDiag.jpg}
\end{figure}}

\only<2> {
\begin{figure}
\centering
\includegraphics[width = 9cm]{images/F1Diag.jpg}
\end{figure}}

\only<3> {
\begin{figure}
\centering
\includegraphics[width = 9cm]{images/IsaacsonDiag.jpg}
\end{figure}}
\end{frame}



\subsection{Comparaison des algorithmes}

\begin{frame}{Comparaison des algorithmes (1)}
\only<1> {
\begin{figure}
\centering
\includegraphics[width = 5cm]{images/Rahn77.jpg} \hspace{0.5cm}
\includegraphics[width = 5cm]{images/RahnDiag.jpg}
\end{figure}}

\only<2> {
\begin{figure}
\centering
\includegraphics[width = 9cm]{images/Rahn77.jpg}
\end{figure}}

\only<3> {
\begin{figure}
\centering
\includegraphics[width = 9cm]{images/RahnDiag.jpg}
\end{figure}}
\end{frame}



\begin{frame}{Comparaison des algorithmes (2)}
\only<1> {
\begin{figure}
\centering
\includegraphics[width = 5cm]{images/Morris77.jpg} \hspace{0.5cm}
\includegraphics[width = 5cm]{images/MorrisDiag.jpg}
\end{figure}}

\only<2> {
\begin{figure}
\centering
\includegraphics[width = 9cm]{images/Morris77.jpg}
\end{figure}}

\only<3> {
\begin{figure}
\centering
\includegraphics[width = 9cm]{images/MorrisDiag.jpg}
\end{figure}}
\end{frame}



\begin{frame}{Comparaison des algorithmes (3)}
\only<1> {
\begin{figure}
\centering
\includegraphics[width = 5cm]{images/RfC77.jpg} \hspace{0.5cm}
\includegraphics[width = 5cm]{images/RfCDiag.jpg}
\end{figure}}

\only<2> {
\begin{figure}
\centering
\includegraphics[width = 9cm]{images/RfC77.jpg}
\end{figure}}

\only<3> {
\begin{figure}
\centering
\includegraphics[width = 9cm]{images/RfCDiag.jpg}
\end{figure}}
\end{frame}


\section*{Fin}

\begin{frame}
\begin{figure}
\centering
\includegraphics[width = 10cm]{images/taf.jpg}
\end{figure}
\end{frame}

\begin{frame}{LZ77: exemple}
\begin{block}{Entrée}
$\texttt{I}=ABCABCABD$
\end{block}

\bigskip

\only<2> {
\begin{tabular}{|c|c|c|c|c|c|c|c|c|c|c|c|c|c|c|}
\hline
\textbf{Étape}&\multicolumn{5}{|c|}{\textbf{Buffer}} & \multicolumn{9}{|c|}{\textbf{Entrée} (\guill{Aperçu})} \\
\hline
0&&&&&&A&B&C&A&B&C&A&B&D\\
\hline
\end{tabular}}

\only<3> {
\begin{tabular}{|c|c|c|c|c|c|c|c|c|c|c|c|c|c|c|}
\hline
\textbf{Étape}&\multicolumn{5}{|c|}{\textbf{Buffer}} & \multicolumn{9}{|c|}{\textbf{Entrée} (\guill{Aperçu})} \\
\hline
0&&&&&&A&B&C&A&B&C&A&B&D\\
\hline
1&&&&&A&B&C&A&B&C&A&B&D&\\
\hline
\end{tabular}}

\only<4> {
\begin{tabular}{|c|c|c|c|c|c|c|c|c|c|c|c|c|c|c|}
\hline
\textbf{Étape}&\multicolumn{5}{|c|}{\textbf{Buffer}} & \multicolumn{9}{|c|}{\textbf{Entrée} (\guill{Aperçu})} \\
\hline
0&&&&&&A&B&C&A&B&C&A&B&D\\
\hline
1&&&&&A&B&C&A&B&C&A&B&D&\\
\hline
2&&&&A&B&C&A&B&C&A&B&D&&\\
\hline
\end{tabular}}

\only<5> {
\begin{tabular}{|c|c|c|c|c|c|c|c|c|c|c|c|c|c|c|}
\hline
\textbf{Étape}&\multicolumn{5}{|c|}{\textbf{Buffer}} & \multicolumn{9}{|c|}{\textbf{Entrée} (\guill{Aperçu})} \\
\hline
0&&&&&&A&B&C&A&B&C&A&B&D\\
\hline
1&&&&&A&B&C&A&B&C&A&B&D&\\
\hline
2&&&&A&B&C&A&B&C&A&B&D&&\\
\hline
3&&&A&B&C&A&B&C&A&B&D&&&\\
\hline
\end{tabular}}

\bigskip

\begin{block}{Sortie}
$\textbf<2>{(0,0,A)}$,
$\textbf<3>{(0,0,B)}$,
$\textbf<4>{(0,0,C)}$,
$\textbf<5>{(3,5,D)}$
\end{block}
\end{frame}

\begin{frame}{LZ77 : décompression}
\begin{block}{Entrée}
$\texttt{I}=ABCABCABD$
\end{block} \bigskip

\begin{tabular}{ll}
Décompression & Compression \\
\uncover<3->{$A$} & $\textbf<2,3>{(0,0,A)}$ \\
\uncover<5->{$B$} & $\textbf<4,5>{(0,0,B)}$ \\
\uncover<7->{$C$} & $\textbf<6,7>{(0,0,C)}$ \\
\uncover<9->{$A$} & $\textbf<8,9,10,11>{(3,5,D)}$ \\
\uncover<9->{$B$} & \\
\uncover<9->{$C$} & \\
\uncover<10->{$A$} & \\
\uncover<10->{$B$} & \\
\uncover<11->{$D$} & \\
\end{tabular}
\end{frame}


\begin{frame}{Matrices}
\begin{figure}
\centering
\only<1>{\includegraphics[width = 7cm]{images/pretty1.jpg}}
\only<2>{\includegraphics[width = 7cm]{images/pretty2.jpg}}
\only<3>{\includegraphics[width = 7cm]{images/pretty3.jpg}}
\only<4>{\includegraphics[width = 7cm]{images/pretty4.jpg}}
\end{figure}
\end{frame}

\end{document}













